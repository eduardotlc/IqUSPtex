% MIT License

% Copyright (c) [2023] [Eduardo Toledo Campos] [eduardotcampos@usp.br]

% Theme created for Univerity of São Paulo Institute of chemistry.

% Images of the file present in the Attachments folder.

% Background logo of the slides can be substituted in the Attachments folder by 
% a desired one, that should be renamed to logo.png or logo.svg

% Thanks to original latex timeline code creator claudio.fiandrino@gmail.com 

% Github https://github.com/cfiandra/timeline tikzlibrarytimeline.code.tex moved
% to the timeline diagrams folder, the github also contains timeline example of 
% latex file to be consulted.

% Chemformula and chemfig packages used in chemistry writings and drawings, can 
% be removed if this 2 elements are not desired.

% Chemistry drawing are created by converting a .mol molecule file by the 
% mol2chemfigPy3 python package, available in pypi and in 
% https://github.com/Augus1999/mol2chemfigPy3

% Thanks to Augus1999 mol2chemfigPy3 creator.

% Citations in Quaderni degli Avogadro Colloquia style.

% Theme headline navigation bar, image caption, footnote and background logo 
% defineds in beameroutertheme file.

% Color palette defined in the beamercolortheme file.

% Font sizes and style defined in beamerfonttheme file.

% Results part removed from the theme, due to not have being published yet.

%--------------------------------------------------------------------------
%--------------------------------------------------------------------------


%##########%
% Packages %
%##########%


\documentclass[aspectratio=169,fleqn,table]{beamer}
\newsavebox{\longestsec}
\mode<presentation>
\usetheme{IqDu}
\setbeamertemplate{navigation symbols}{}
\pdfpageattr{/Group <</S /Transparency /I true /CS /DeviceRGB>>}
\usepackage{arevmath}
\usepackage{booktabs}
\usepackage{etoolbox}
\usepackage{helvet}
\usepackage{xcolor}
\usepackage{array}
\usepackage{graphicx}
\usepackage{nicefrac}
\usepackage{chemformula}
\usepackage{tikz}
\usepackage[T1]{fontenc}
\usepackage[utf8]{inputenc}
\usepackage{babel}

\begin{document}


%#######%
% Title %
%#######%


\begin{frame}[plain]
	
    \titlepage
  
\end{frame}


%#######%
% Index %
%#######%


\begin{frame}[plain]
\frametitle{Index}

\hfill
\parbox[t]{.55\textwidth}{
    \scriptsize
    \begin{minipage}[c][0.4\textheight]{\textwidth}
    \tableofcontents
    \end{minipage}
    
}
\end{frame}

%##############%
% Introduction %
%##############%


\section{Introduction}


% Motivation
\subsection{Motivations}
\begin{frame}
\frametitle{Motivations}

\begin{itemize}
    \item
    Numerous applications of singlet oxygen
    \item
    Active development research area
    \item
    Necessity of new sensitizer composites development, to overcome boundries of the area
\end{itemize}

\end{frame}


% Project 
\subsection {Project}
\begin{frame}[plain]
\frametitle{Project}

\vspace{-35pt}
\begin{figure}[border=1pt]
	\pgfimage[width=0.95\textwidth]{Project/project.pdf}
\end{figure}

\end{frame}

% Insertion of a big image, with de definition
% of 4 zoom boxes, for better visualization of 
% the image subareas

% Reactive oxygen species
\subsection{Reactive oxygen species}
\begin{frame}<1>[label=zooms]
\frametitle<1>{Reactive oxygen species}
    
\framezoom<1><2>(2cm,-0.1cm)(1.3cm,2.95cm)
\framezoom<1><3>(2cm,3cm)(1.3cm,2.95cm)
\framezoom<1><4>(6.5cm,0.1cm)(0.7cm,2.6cm)
\framezoom<1><5>(6.5cm,2.7cm)(1cm,2.7cm)
\begin{center}
    \vspace{-14pt}
    \pgfimage[width=0.7\textwidth]{Attachments/ROS_Quali.png}
    \vspace{-10pt}
\end{center}
    
\end{frame}
\againframe<2->[plain]{zooms}


% Singlet oxygen
\subsection{Singlet oxygen}

\begingroup
\setbeamertemplate{caption}{%
\begin{beamercolorbox}[wd=.2\paperwidth, sep=.2ex]{block body}\insertcaption%
\end{beamercolorbox}%
}
\begin{frame}
\frametitle{Singlet oxygen ($\ch{^1$\Delta$_g}$)}

\begin{figure}\centering
\begin{minipage}[b]{0.46\textwidth}
	\centering
	\vspace{-10pt}
	\pgfimage[width=1\textwidth]{Attachments/GO_OD.png}
    \vspace{-5pt}
	\caption{%
    \textit{Groundstate $\ch{^3$\Sigma${{_g}{^-}}}$}}
\end{minipage}
\hfill
\begin{minipage}[b]{0.46\textwidth}
	\centering
	\vspace{-10pt}
	\pgfimage[width=1\textwidth]{Attachments/SO_OD.png}
    \vspace{-5pt}
	\caption{%
    \textit{Excited state  $\ch{^1$\Delta$_g}$}}
\end{minipage}
\end{figure}

\end{frame}
\endgroup


% Photosensitized oxidation reactions type I and II
\begingroup
\setbeamertemplate{footline}{}
\subsection{Photosensitized oxidation reactions}
\begin{frame}
\blfootnote{[1] M.S. Baptista et al., Photochemistry and Photobiology, 2017, 93, 912.}
\frametitle{Photosensitized oxidation reactions type I and II}
  
\begin{itemize}
	\item
	Have oxygen as reactant
    \item
    Abstraction of one elctron or hydrogen atom as oxidizing step
    \item
    $\ch{O2}$ participates in one of the following ways:
    \begin{enumerate}
    	\item
    	Directly, as one electron oxidizer
        \item
        Indirectly, by the generation of $\ch{O2^{.-}}$
    \end{enumerate}
\end{itemize}
    
\end{frame}
\endgroup


% Photosensitized oxidation reactions type I and II
\begingroup
\setbeamertemplate{footline}{}
\begin{frame}
\blfootnote{[1] M.S. Baptista et al., Photochemistry and Photobiology, 2017, 93, 912.}
\frametitle{Photosensitized oxidation reactions type I and II}

\begin{figure}
\centering
\begin{minipage}[border=15pt]{0.46\textwidth}
	\centering
	\vspace{-175pt}
    \begin{enumerate}
        \item
        Type I :
        \begin{itemize}
            \item
            Photoinduced electron transfer
            \item
            Formation of $\ch{O}$ e $\ch{HO2^.}$
        \end{itemize}
        \item
        Type II :
        \begin{itemize}
            \item
            Sensibilized formation of $\ch{^1O2}$
            \item
            Energy transfer $sensibilizer\ch{-> O2}$
        \end{itemize}
    \end{enumerate}
    \vspace{-10pt}
    \mybeamerthm{pdtypes}{}
    \begin{pdtypes}
        Photodynamic action refers to apoptosis by mechanisms I and II
    \end{pdtypes}
\end{minipage}
\hfill
\begin{minipage}[b]{0.5\textwidth}
	\centering
	\vspace{-10pt}
	\pgfimage[width=1.1\textwidth]{Attachments/type1.png}
    \vspace{-5pt}
    \caption {%
    \textit{Example of type I photosensibilized oxidation}}
\end{minipage}
\end{figure}

\end{frame}
\endgroup


% Photodynamic therapy - PDT
\subsection{Photodynamic therapy - PDT}
\begingroup
\setbeamertemplate{footline}{}
\begin{frame}
\blfootnote{[2] Thomas J. Dougherty, Outskirt press inc, 2015.}
\frametitle{Photodynamic therapy - PDT}

\begin{figure}
\centering
\begin{minipage}[border=15pt]{0.45\textwidth}
	\centering
	\vspace{-165pt}
    \begin{itemize}
        \item
        Photosensibilizing agent
        \begin{itemize}
            \item
            Topic administrated or injected
        \end{itemize}
        \item
        Light
        \item
        Aims destruction of abnormal cells
        \item
        3 steps process:
        \begin{enumerate}
            \item{Administration of photosensibilizer drug}
            \item{Drug activation by light}
            \item{Targeted cells elimination}
        \end{enumerate}
    \end{itemize}
\end{minipage}
\hfill
\begin{minipage}[b]{0.54\textwidth}
	\centering
	\vspace{-10pt}
	\pgfimage[width=1\textwidth]{Attachments/pdt.png}
    \caption {%
    \textit{Example of tumour treatment by PDT}}
\end{minipage}
\end{figure}

\end{frame}
\endgroup


% Timeline
\begin{frame}[plain]
\frametitle{PDT development}

\vspace{-35pt}
\begin{figure}[border=1pt]
	\pgfimage[width=0.95\textwidth]{Timeline/example.pdf}
\end{figure}

\end{frame}


% Chemodynamic therapy - CDT
\subsection{Chemodynamic therapy - CDT}
\begin{frame}
\frametitle{Chemodynamic therapy - CDT}

\begin{itemize}
	\item
	ROS generated by chemical reaction
	\item
	Doesn't involve light excitation
	\item
	Chemical agent is activated by:
	\begin{itemize}
    	\item
        Tumour microenvironment	
		\begin{enumerate}
			\item
			Low pH
			\item
			High concentration of specific ions
		\end{enumerate}
		\item
		External stimulus
		\begin{enumerate}
			\item
			Heat
			\item
			Ultrasound
		\end{enumerate}
	\end{itemize}
	\item
	MOFs have important applications
\end{itemize}     	      
		      
\end{frame}


% PDT advantages
\subsection{PDT advantages}
\begin{frame}
\frametitle{PDT advantages}

\begin{enumerate}
	\item
	Controled activation
	\item
	Overcome resistance mechanism 
	\item
	Non invasive treatment
	\item
	Intensisty control
	\item
	High specificity
\end{enumerate}

\end{frame}


% PDT disadvantages
\subsection{PDT disadvantages}
\begin{frame}
\frametitle{PDT disadvantages}

\begin{enumerate}
	\item\textbf{%
	Limited light penetration in biological tissues}
	\item
	Caution needed with sunlight and intense light sources after treatment
	\item\textbf{%
	Low $\ch{O2}$ concentration at the tumour microenvironment}
\end{enumerate}

\end{frame}


%###############%
% Nanoparticles %
%###############%


\section{Nanoparticles}


% Reversible binding of oxygen
\subsection{Reversible binding of oxygen}
\begin{frame}
\frametitle{Nanoparticles}
\framesubtitle{Reversible binding of oxygen}

\vspace{-25pt}
\begin{figure}[border=1pt]
	\pgfimage[width=0.85\textwidth]{Attachments/NP_RB.png}
\end{figure}	

\end{frame}


% Plasmonic
\subsection{Plasmonic}
\begin{frame}
\frametitle{Nanoparticles}
\framesubtitle{Plasmonic}

\begin{figure}
\centering
\begin{minipage}[border=15pt]{0.45\textwidth}
	\centering
	\vspace{-165pt}
	\begin{itemize}
		\item
		Study of behavior and interaction of \textbf{plasmons}
		\mybeamerthm{plsm}{\textbf{Plasmons}}
		\begin{plsm}
			\textbf{Colective oscillation of electrons in a metallic structure , generated by electromagnetic waves}
		\end{plsm}
		\item
		Applications in catalysis, photonics, imaging, sensors, energy conversion, etc...
	\end{itemize}
\end{minipage}
\hfill
\begin{minipage}[b]{0.5\textwidth}
	\centering
	\vspace{-25pt}
	\pgfimage[width=1\textwidth]{Attachments/sine_plasmonic_wave.png}
    \caption {%
    \textit{Oscilation of a nanoprticle field, under the influence of a electromagnetic wave}}
\end{minipage}
\end{figure}

\end{frame}


% Plasmonic
\begin{frame}
\frametitle{Nanoparticles}
\framesubtitle{Plasmonic}

\begin{enumerate}
	\item
	Surface plamon ressonance (SPR)
	\begin{itemize}
		\item
		Absorption
		\item
		Diffraction
		\item
		Transmission
	\end{itemize}
	\item
	Hot electrons
\end{enumerate}
\begin{itemize}
    \item
    Light interaction depends on the material design
\end{itemize}

\end{frame}


% Upconversion
\subsection{Upconversion}
\begingroup
\setbeamertemplate{footline}{}
\begin{frame}
\blfootnote{[3] I.P. Machado et al., Journal of Alloys and Compounds, 2023, 942, 169083.}
\frametitle{Nanoparticles}
\framesubtitle{Upconversion}

\begin{figure}
\centering
\begin{minipage}[border=15pt]{0.45\textwidth}
  	\centering
  	\vspace{-180pt}
  	\begin{itemize}
  	\mybeamerthm{uc}{\textbf{Upconversion}}
  	\begin{uc}
  		\textbf{Low energy photons conversion process, tipically in the near-infrared, to higher energy photons at the UV-VIS region}
  	\end{uc}
  		\item
  		Rare earth nanoparticles
		\item
		May be optimized by plasmonic nanoaprticles
  	\end{itemize}
\end{minipage}
\hfill
\begin{minipage}[b]{0.45\textwidth}
	\centering
	\vspace{-40pt}
	\pgfimage[width=0.6\textwidth]{Attachments/np_uc_lucas.png}
	\caption{%
	\textit{$\ch{Gd_2O_2S:Er^{3+},Yb^{3+}}$ emission under 980 nm irradiation}}
\end{minipage}
\end{figure}

\end{frame}
\endgroup


% Upconversion
\begingroup
\setbeamertemplate{footline}{}
\begin{frame}
\blfootnote{[4] X. Bai et al., ACS Applied Materials \& Interfaces, 2020, 12, 21936.}
\frametitle{Upconversion}
  
\begin{figure}
\centering
\begin{minipage}[border=15pt]{0.5\textwidth}
	\centering
	\vspace{-185pt}
	\begin{enumerate}
		\item
		980 nm excitation
		\item
		$\ch{^4I_{11/2}}$ e $\ch{^2F_{5/2}(Yb^{3+})}$ ressonance
		\item
		\textbf{ESA:} Multiphotons excited state absorption
		\item
		\textbf{ETU:} $\ch{Er^{3+}}$ excitation and $\ch{Yb^{3+}}$ non-radiative relaxation
		\item
		Multifonon relaxation
	\end{enumerate}
\end{minipage}
\hfill
\begin{minipage}[b]{0.4\textwidth}
	\centering
	\vspace{-35pt}
	\pgfimage[width=1.1\textwidth]{Attachments/UC_NP.png}
    \vspace{-6pt}
    \caption{%
    \textit{Rare earth crystal upconversion energy diagram}}
\end{minipage}
\end{figure}

\end{frame}
\endgroup


%#####
% MOFs
%#####


\section{MOFs}


% Introduction MOF
\begingroup
\setbeamertemplate{caption}{%
\begin{beamercolorbox}[wd=.13\paperwidth, sep=.2ex]{block body}\insertcaption%
\end{beamercolorbox}%
}
\begin{frame}
\frametitle{MOFs}

\begin{figure}
\centering
\begin{minipage}[border=15pt]{0.47\textwidth}
	\centering
	\vspace{-175pt}
    \begin{itemize}
        \item
        Porous materials
        \item
        Cristallyne
        \item
        Three-dimensional structured
        \item
        \textbf
        Metallic centers coordinated to \textbf{organic ligands}
    \end{itemize}
\end{minipage}
\hfill
\begin{minipage}[b]{0.5\textwidth}
	\centering
	\vspace{-35pt}
	\pgfimage[width=1\textwidth]{Attachments/zif8.png}
    \caption{%
    \textit{ZIF-8 structure}}
\end{minipage}
\end{figure}

\end{frame}
\endgroup


% Introduction MOF
\begin{frame}
\frametitle{MOFs}

\begin{itemize}
    \item{Vantagens}
    \begin{enumerate}
        \item
        High porosity
        \item
        Structural versatility
        \item
        Modullable properties
        \item
        High sustentabillity
    \end{enumerate}
    \item{Aplications}
    \begin{enumerate}
        \item
        Gas storage and separation
        \item
        Catalysis
        \item
        Sensors/detection
        \item
        Drug delivery
        \item
        Energy
    \end{enumerate}
\end{itemize}

\end{frame}


% Reversible oxygen binding
\subsection{Reversible oxygen binding}
\begingroup
\setbeamertemplate{caption}{%
\begin{beamercolorbox}[wd=.25\paperwidth, sep=.2ex]{block body}\insertcaption%
\end{beamercolorbox}%
}
\setbeamertemplate{footline}{}
\begin{frame}

\blfootnote{[5] J. Park et al., Angewandte Chemie International Edition, 2015, 54, 430.}
\frametitle{MOFs}
\framesubtitle{Reversible oxygen binding}
\begin{figure}
\centering
\begin{minipage}[border=15pt]{0.45\textwidth}
    \vspace{-160pt}
    \begin{itemize}
        \item
        Overcome hipoxia conditions in tumour microenvirmoments
        \item
        Oxygen adsorption
        \item
        Also presents catalysis applications, sensors and gas separations
    \end{itemize}
\end{minipage}
\hfill
\begin{minipage}[b]{0.4\textwidth}
	\centering
	\vspace{-35pt}
	\pgfimage[width=0.8\textwidth]{Attachments/rvsmof.png}
    \vspace{-6pt}
    \caption{%
    \textit{MOF SO-PCN structure}}
\end{minipage}
\end{figure}

\end{frame}
\endgroup


% Upconversion
\subsection{Upconversion}
\begin{frame}
\frametitle{MOFs}
\framesubtitle{Upconversion}

\begin{itemize}
    \item
    Classical mechanism on rare earth metallic center MOFs
    \item
    \textbf{Triplet-triplet annihilation (TTA)}
    \item
    Organic ligand acting as sensitizer
\end{itemize}

\end{frame}


% Upconversion
\begin{frame}
\frametitle{MOFs}
\framesubtitle{Upconversion}
\begin{enumerate}
    \item
    Low energy photon absorption by the organic ligand
    \item
    Excitation to long-lived triplet state
    \item
    Energy transfer to a triplet state of an "annihilator" molecule
    \item
    "Annihilator" transfer to another close annihilator
    \item
    Decay of both molecules to a singlet state
    \item
    Upconversion photon emission
\end{enumerate}

\end{frame}
% PROJETO

\section{Project}


% Project
\begin{frame}[plain]
\frametitle{Project}

\vspace{-35pt}
\begin{figure}[border=1pt]
	\pgfimage[width=0.95\textwidth]{Project/project.pdf}
\end{figure}

\end{frame}



% SYNTHESIS

\section{Synthesis}


\subsection{MOF $\ch{Zn2(SDC)2(An2Py)}$}

% MOF Zn2(SDC)2(An2Py)
\begin{frame}
\frametitle{MOF $\ch{Zn2(SDC)2(An2Py)}$}
\framesubtitle{Anthracene Dicarbaldehyde($\ch{AnAd2}$)}

\vspace{-35pt}
\begin{figure}[border=1pt]
	\pgfimage[width=1.08\textwidth]{Fluxograms/anad2.pdf}
\end{figure}

\end{frame}


% MOF Zn2(SDC)2(An2Py
\begin{frame}
        \frametitle{MOF $\ch{Zn2(SDC)2(An2Py)}$}
        \framesubtitle{Anthracene dicarbaldehyde ($\ch{AnAd2}$)}

        \begin{figure}\centering
                \begin{minipage}[b]{0.46\textwidth}
                        \centering
                        \vspace{-10pt}
                        \pgfimage[width=1\textwidth]{Attachments/anad21.jpg}
                        \vspace{-5pt}
                        \caption{%
                                \textit{\ \ \ \ \  System to $\ch{AnAd2}$ synthesis}}
                \end{minipage}
                \hfill
                \begin{minipage}[b]{0.46\textwidth}
                        \centering
                        \vspace{-10pt}
                        \pgfimage[width=1\textwidth]{Attachments/anad22.jpg}
                        \vspace{-5pt}
                        \caption{%
                                \textit{\ \ \ \ \ \ \ \ \ \ \ \ \ \ \ \ \ \ \ \ \ \ \ $\ch{AnAd2}$ synthesis}}
                \end{minipage}
        \end{figure}

\end{frame}


% MOF Zn2(SDC)2(An2Py)
\begin{frame}
\frametitle{MOF $\ch{Zn2(SDC)2(An2Py)}$}
\framesubtitle{9,10-Bis[2-(4-pyridyl)vinyl]anthracene(An2Py)}

\vspace{-35pt}
\begin{figure}[border=1pt]
	\pgfimage[width=1.08\textwidth]{Fluxograms/an2py.pdf}
\end{figure}

\end{frame}


%Pd@Ag
\subsection{Pd@Ag}
\begin{frame}
\frametitle{Nanoparticles}
\framesubtitle{Pd@Ag}
\begin{figure}
    \centering
    \vspace{-40pt}
    \begin{minipage}[b]{0.46\textwidth}
        \vspace{-15pt}
        \begin{table}
            \vspace{-140pt}
            \hspace{-20pt}
            \begin{tabular}
                {|>{\columncolor{lightnormalblue}}c |>{\columncolor{lightfadingblue}}c | l | l | l | l | l | l | l | l | l |}
                \toprule
                \textbf{Pd@Ag} & \textbf{Length} \\
                \hline
                count & 361    \\
                \hline
                mean  & 18.3   \\
                \hline
                std   & 2.2    \\
                \hline
                min   & 12.1   \\
                \hline
                25\%  & 16.6   \\
                \hline
                50\%  & 18.1   \\
                \hline
                75\%  & 19.7   \\
                \hline
                max   & 26.7   \\
                \bottomrule
            \end{tabular}
        \end{table}
    \end{minipage}
    \hfill
    \begin{minipage}[b]{0.52\textwidth}
        \centering
        \vspace{30pt}
        \pgfimage[width=1.1\textwidth]{Attachments/pdaghist.png}
        \vspace{-6pt}
        \caption{%
            \textit{Nanoparticles size distribution}}
    \end{minipage}
\end{figure}
\end{frame}

% CONCLUSION
\section{Conclusion}


% Next steps
\subsection{Next steps}
\begin{frame}
  \frametitle{Next steps}
  % Content for Subsection 6.1
\end{frame}


% Conclusão
\subsection{Conclusion}
\begin{frame}
  \frametitle{Conclusion}
  % Content for Subsection 6.2
\end{frame}

\end{document}
